% THIS IS SIGPROC-SP.TEX - VERSION 3.1
% WORKS WITH V3.2SP OF ACM_PROC_ARTICLE-SP.CLS
% APRIL 2009
%
% It is an example file showing how to use the 'acm_proc_article-sp.cls' V3.2SP
% LaTeX2e document class file for Conference Proceedings submissions.
% ----------------------------------------------------------------------------------------------------------------
% This .tex file (and associated .cls V3.2SP) *DOES NOT* produce:
%       1) The Permission Statement
%       2) The Conference (location) Info information
%       3) The Copyright Line with ACM data
%       4) Page numbering
% ---------------------------------------------------------------------------------------------------------------
% It is an example which *does* use the .bib file (from which the .bbl file
% is produced).
% REMEMBER HOWEVER: After having produced the .bbl file,
% and prior to final submission,
% you need to 'insert'  your .bbl file into your source .tex file so as to provide
% ONE 'self-contained' source file.
%
% Questions regarding SIGS should be sent to
% Adrienne Griscti ---> griscti@acm.org
%
% Questions/suggestions regarding the guidelines, .tex and .cls files, etc. to
% Gerald Murray ---> murray@hq.acm.org
%
% For tracking purposes - this is V3.1SP - APRIL 2009

\documentclass{acm_proc_article-sp}

\begin{document}

\title{On the Ground Validation of Online Diagnosis with Twitter and Medical Records}


%
% You need the command \numberofauthors to handle the 'placement
% and alignment' of the authors beneath the title.
%
% For aesthetic reasons, we recommend 'three authors at a time'
% i.e. three 'name/affiliation blocks' be placed beneath the title.
%
% NOTE: You are NOT restricted in how many 'rows' of
% "name/affiliations" may appear. We just ask that you restrict
% the number of 'columns' to three.
%
% Because of the available 'opening page real-estate'
% we ask you to refrain from putting more than six authors
% (two rows with three columns) beneath the article title.
% More than six makes the first-page appear very cluttered indeed.
%
% Use the \alignauthor commands to handle the names
% and affiliations for an 'aesthetic maximum' of six authors.
% Add names, affiliations, addresses for
% the seventh etc. author(s) as the argument for the
% \additionalauthors command.
% These 'additional authors' will be output/set for you
% without further effort on your part as the last section in
% the body of your article BEFORE References or any Appendices.

\numberofauthors{3} %  in this sample file, there are a *total*
% of EIGHT authors. SIX appear on the 'first-page' (for formatting
% reasons) and the remaining two appear in the \additionalauthors section.
%
\author{
% You can go ahead and credit any number of authors here,
% e.g. one 'row of three' or two rows (consisting of one row of three
% and a second row of one, two or three).
%
% The command \alignauthor (no curly braces needed) should
% precede each author name, affiliation/snail-mail address and
% e-mail address. Additionally, tag each line of
% affiliation/address with \affaddr, and tag the
% e-mail address with \email.
%
% 1st. author
\alignauthor
Todd Bodnar\titlenote{Corresponding author}\\
       \affaddr{Pennsylvania State University}\\
\affaddr{Department of Biology}\\
       \affaddr{University Park, PA 16802}\\
       \email{tjb5215@psu.edu}
% 2nd. author
\alignauthor
Maybe Conrad, Maybe Vicky
% 3rd. author
\alignauthor
Marcel Salath\'e\\
       \affaddr{Pennsylvania State University}\\
\affaddr{Department of Biology}\\
       \affaddr{University Park, PA 16802}\\
       \email{salathe@psu.edu}
}

\maketitle
\begin{abstract}
This is an abstract
\end{abstract}

% A category with the (minimum) three required fields
%\category{I.6.4}{Simulation and Modeling}{Model Validation and Analysis}
\category{I.2.1}{Artificial Intelligence}{Applications and Expert Systems}[Medicine and Science]
%A category including the fourth, optional field follows...
%\category{D.2.8}{Software Engineering}{Metrics}[complexity measures, performance measures]

\terms{Experimentation, Validation}

\keywords{Twitter, Validation, Digital Epidemiology, Remote Diagnosis} % NOT required for Proceedings

\section{Introduction}

\section{Related Work}

\section{Data Collection}
\subsection{Medical Records}
\subsection{Twitter Records}

\section{Signal Detection}
\subsection{Event Based Signals}
\subsection{Frequency Based Signals}

\begin{figure}
\centering
\includegraphics[width=0.5\textwidth]{figs/meanFrequencies.eps}
\caption{The frequency of tweeting behaviour of individuals in the months before, during and after an illness. Users significantly (check) decrease their rate of tweeting during the time that they had influenza. Dashed lines indicate the mean rate for the three months before / after the illness. (Todo: check significance)}
\label{fig:mean_freq}
\end{figure}


\subsection{Network Based Signals}

\section{Analysis}

\section{Conclusions}

\begin{thebibliography}{1}
\bibitem{Ford:1956vc}
L. R. Ford and D. R. Fulkerson. \newblock{Maximal Flow through a Network}. \newblock{\em Canadian Journal of Mathematics}, 8(3):399-404, 1956.
\end{thebibliography}



%@article{Ford:1956vc,
%author = {Ford, L R and Fulkerson, D R},
%title = {{Maximal Flow through a Network.}},
%journal = {Canadian Journal of Mathematics},
%year = {1956},
%pages = {399--404}
%}


%
% The following two commands are all you need in the
% initial runs of your .tex file to
% produce the bibliography for the citations in your paper.
\bibliographystyle{abbrv}
\bibliography{sigproc}  % sigproc.bib is the name of the Bibliography in this case
% You must have a proper ".bib" file
%  and remember to run:
% latex bibtex latex latex
% to resolve all references
%
% ACM needs 'a single self-contained file'!
%
%APPENDICES are optional
%\balancecolumns

\balancecolumns
% That's all folks!
\end{document}
